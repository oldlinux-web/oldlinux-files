%
% A small Linux paper written partly because I need to get
% my LaTeX going :-), and partly because a paper is a good idea.
%
% NOTE! This is *La*TeX, not ordinary TeX
%

\def\nic{{\tt `nic.funet.fi'}}
\def\dir{{\tt '/pub/OS/Linux'}}
\def\ftp{{\tt ftp}}
\def\Linux{\begingroup\sc Linux\endgroup}

\documentstyle{article}
\title{\Linux---a free unix-386 kernel}
\author{Linus Torvalds (torvalds@kruuna.helsinki.fi)}
\date{October 10, 1991}
\begin{document}
\maketitle

\section{Introduction to \Linux}

\subsection{What is \Linux?}

\Linux\ is a free unix-like kernel for 386-AT computers, coming with
full source code. It is meant for hackers/computer science students to
use, learn and enjoy. It is written mostly in {\tt\bf C}, but parts of
it are in gnu-format assembler, and the boot-sequence is in intel 086
assembly language. The {\tt\bf C}-code is relatively ANSI, with a few
GNU enhancements (mostly {\tt\_\_asm\_\_} and {\tt inline}).

While there are a number of unices available for 386 computers, most of
them cost a lot of money, and come with no source. Thus, they are ideal
for actually {\it using} your computer, but if you want to learn how
they work, you are f--\,--ked.

There are also a few unices available with source. Minix, the learning
tool written by Andrew S.~Tanenbaum, has been used in universities as a
teaching tool for years. The BSD-386 system comes with source, but has a
restrictive copyright and costs a lot of money (\$995 is the starting
price, I think). The GNU kernel (Hurd) will be free, but is currently
not ready, and will be too big to understand and learn.

\Linux\ most closely resembles Minix, in that it is small and not very
sophisticated, and thus easy (well\dots) to understand.  \Linux\ was
also written under Minix, so there are quite a bit of similarities, and
any Minix hacker will feel relatively at home with \Linux.  None of the
Minix code was used in the project though, so the Minix copyright
doesn't cover the new system.  It also is {\it completely free}, and has
a very loose copyright.  Thus there is no need for megabytes of diffs
like under Minix. 

\subsection{The \Linux\ copyright}

While being freely distributable, I do restrict the use of \Linux\ in a
few ways:
%
\begin{itemize}
\item You may freely copy and redistribute the source and binaries, {\bf
as long as:}
\begin{itemize}
\item complete source is available. Thus binaries may not be distributed
by themselves, even if you have made changes to them.
\item you do not profit from the distribution. In fact even ``handling
costs'' are not acceptable.
\item you keep the appropriate copyrights intact.
\end{itemize}
\item You may change the source to your liking, but if you distribute
parts of the new system (or just binaries), all the new code must be
included.
\item You may make small excepts from the code without including
copyrights. This is up to you, but a reference to me or the code would
be appreciated.
\end{itemize}
This should be loose enough not to cause any worry in anybody using or
expanding the system. If you have a friend who really doesn't want the
source, but just a working binary, you may of course give it to him
without worrying whether I will sue you. Keep it between friends, though.

\subsection{Hardware/software needed to get \Linux\ running}

\Linux\ was written on a 386-AT running Minix. As \Linux\ is a real
operating system, and goes directly to the hardware to do things, you
have to have a very similar system to get it going without problems:
\begin{itemize}
\item 386-AT (PS/2's are different enough that things won't work)
\item VGA or EGA screen hardware.
\item Standard AT hard-disk interface, IDE disks work fine (in fact
that's what I use).
\item Normal real-mode BIOS. Some machines seem to use virtual-86 mode
to run the bootup program, and on such machines \Linux\ won't boot up
and run correctly.
\end{itemize}

While \Linux\ will be expanded to be a self-sufficient system, Minix-386
is currently needed to get the ball rolling. You need Minix to make the
initial root file system, and to compile the OS binary. After that
\Linux\ is a self-sufficient system, but Minix is recommended in order
to make file system checking ({\tt fsck}) and to recompile the system
after making changes.

\subsection{Getting \Linux}

\Linux\ can currently be gotten by anonymous \ftp\ from \nic\ in the
directory \dir. This directory contains the full source to the operating
system, as well as a couple of binaries so that you can actually use the
system.

{\vspace{\baselineskip}\bf NOTE! The binaries are mostly GNU software,
and are under a stricter copyright (the GNU copyleft) than the \Linux\
sources.  Thus you may not redistribute them without distributing their
source, found in {\tt/pub/GNU}. See any GNU software package for more
information on the GNU copyleft.\vspace{\baselineskip}}

The various files found in this directory are:
\begin{itemize}
\item{\tt linux-0.03.tar.Z}---complete source to the operating system,
in a 16-bit compressed tar archive.
\item{\tt Linux.tex}---\LaTeX\ source for this file.
\item{\tt bash.Z}---bash binary to run under \Linux. This binary should
be put under tha name {\tt/bin/sh} in the file system reserved for
\Linux\ (see installation).
\item{\tt update.Z}---update binary, to be put in {\tt/bin/update}.
\item{\tt gccbin.tar.Z}---GNU cc binaries needed to get a working
compiler. This tarred archive contains the compiler, loader, assembler
and support programs (nm, strip etc). It also contains a small library
sufficient for most programs.
\item{\tt include.tar.Z}---include-files necessary to get gcc working.
\item{\tt unistd.tar.Z}---source to the unistd library routines (ie
system call interface). With this you can build a bigger library by
using system-independent library sources.
\item{\tt utilbin.tar.Z}---binaries to various GNU utilities, including
GNU fileutils, make and tar. Also contains the emacs-clone {\tt uemacs}.
\item{\tt README, RELNOTES-0.01, INSTALLATION}---ascii files containing
some (somewhat out-of-date) information about \Linux.
\end{itemize}

The absolute minimum needed to get a system going is the OS source and
the bash and update binaries. You won't be doing much with just these
though.

\section{Installation}

After you have gotten the necessary \Linux\ files, you need to compile
the system and make a root directory. The necessary binaries need to be
put in the root file system.
Do this:
\begin{enumerate}
%
\item Back up your software. While \Linux\ never has destroyed any of my
files, nothing is certain. Better safe than sorry.
%
\item Choose/make a standard Minix HD-partition to be the new \Linux\ root
file system.
%
\item Make the necessary device nodes on the new root.  \Linux\ uses the
same type of nodes as Minix, so use the Minix {\tt mknod} command to
make the following devices:
\begin{itemize}
\item{\tt/dev/tty}
\item{\tt/dev/tty[0-2]}
\item{\tt/dev/hd[0-9]}
\end{itemize}
Node numbers are the same as in Minix.
%
\item Move the necessary files to the new root partition. The files
should be in the following directories:
\begin{itemize}
\item{\tt/bin:}
\begin{itemize}
\item{\tt sh}, ie {\tt bash.Z}.
\item{\tt update}
\end{itemize}
\item{\tt/usr/bin:}
\begin{itemize}
\item Contents of {\tt utilbin.tar.Z}
\end{itemize}
\item{\tt/usr/include:}
\begin{itemize}
\item Contents of {\tt include.tar.Z}
\end{itemize}
\item{\tt/usr/local/lib:}
\begin{itemize}
\item Contents of {\tt gccbin.tar.Z}, excepting {\tt gcc}
\end{itemize}
\item{\tt/usr/local/bin:}
\begin{itemize}
\item{\tt gcc}
\item links to files in {\tt/usr/local/lib} of your choice. I link
{\tt ld}, {\tt as}, {\tt nm}, {\tt strip} and {\tt size} to their
counterparts in {\tt/usr/local/lib/gcc-XXX}.
\end{itemize}
%
\item Edit the {\tt linux/include/linux/config.h} file for your system. 
This file contains the system-specific information: memory space, disk
types, root partition number (again the numbering is the same as in
Minix), keyboard type (currently just US and Finnish) etc. 
%
\item Compile the \Linux\ sources.  A simple {\tt make} should do the
trick, after you have edited the {\tt makefile}s to suit your system
(ie, removed the {\tt-mstring-insns} flag and changed the paths to suit
you.) Users of {\tt gcc} versions earlier than 1.40 will probably have
to add {\tt gnulib} to the `{\tt LIBS =}'-line in the {\tt makefile}. 
%
\item Copy the resultant {\tt Image}-file to a floppy (ie, {\tt cp Image
/dev/PS0} or similar.)
%
\item Reboot with the new floppy. The startup screen should tell you
that the system is booting ({\tt Loading system ...}), then some vital
root file system information ({\tt xxx/XXX inodes/blocks free}),
followed by a {\tt Ok.} and the bash prompt (initially {\tt bash\#} if
you have no {\tt.bashrc} file).
\end{itemize}

Hopefully you now have a functioning unix, and you are logged in as
{\tt root}. \Linux\ currently has no {\tt'init'} process, and as soon
as you log out, the system will {\tt sync} and just wait. Use the
three-finger-salute to reboot your machine.

\end{enumerate}

\section{Things missing/incomplete in \Linux}

While \Linux\ is meant to be a fully selfsufficient kernel, this is
currently not the case. As already mentioned, you need Minix to set
things up, and to check the file system once it's running. There are a
number of other deficiencies:
\begin{itemize}
%
\item Incomplete hardware-support. Some of the standard features of an
AT are not currently supported. Most notably are floppy disk drives,
making using \Linux\ for real work (backing up etc) currently not
possible. Also some of the features of the serial lines aren't yet
implemented (hard-wired to 2400bps, no hang-up notification etc).
%
\item Incomplete standard C library. The {\tt libc.a} found in the gcc
distribution package is not complete, and I'm very much interested in
freely distributable library functions.
%
\item Some of the system calls are not fully implemented.  This concerns
mostly ``seldom-used'' features like debugging (yeah, who needs it
anyway, don't all your programs work the first time {\tt:-)} and some
other features. 
%
\item As mentioned, no {\tt login} and {\tt init} processes. Currently
\Linux\ boots up in single-user mode, with the root as console-user.
This is enough for some porting work, but not really practical.
%
\item 387-support is not yet implemented, although some skeleton
routines are present. The gcc-binary found on \nic\ will correctly use
soft-float (ie emulation function calls) for the four basic math
operations. 387-support will materialize as soon as a 387 finds its way
into my computer. Hopefully in a month or two.
%
\item None of the important system-administration commands has yet been
written for \Linux. These include things like {\tt mkfs}, {\tt format},
{\tt fsck}, {\tt mknod} etc. Some of these need kernel features not yet
implemented ({\tt format}, {\tt mknod}), some just need to be written.
As with the library, I'd welcome any freely distributable files.
%
\end{itemize}

As you can see, \Linux\ is as yet not a complete system.  Your help is
appreciated to make it better.  I'm not interested in Minix-commands
rewritten for \Linux, unless you have written them yourself from
scratch.  You are of course free (and encouraged) to use everything you
have in your Minix-distribution for your own \Linux-system, but due to
the Minix copyrights, they cannot be distributed to a wider audience.

Some of the probelms mentioned here will be fixed by me (ie serial
lines/387/floppy support) as soon as possible, but I'm hoping to get
help with the libraries etc.  Bug-reports/patches and wish-lists will be
appreciated, and if you actually have the patch to the problem, I'll try
to implement it right away. Small changes will be sent out as patches to
the mailing list and be set up on \nic, but after heavy rewrites or
bigger patches, the whole system will be updated at \nic.

\section{Porting software for \Linux}

\Linux\ was designed to make porting relatively easy. Thus the full
{\tt termios}-implementation, and the somewhat {\tt POSIX} library. The
(admittedly relativly few) programs I've ported posed no problems.

Even though \Linux\ resembles Minix a great deal, Minix programs are not
generally easier to port than programs designed for some other unix.
Thus I wouldn't recommend starting from a Minix-version of a particular
program, but instead trying to port the ``virgin'' program from scratch.
Being closer to SYSV than BSD means that most programs port easily when
given a {\tt-DUSG} or {\tt-DSYSV} flag.

One of the more difficult problem in porting can be missing library
functions. These must be written by you, or copied from some other
source (Minix being one possibility for those that have it).
Alternatively, some programs (notably GNU) have various flags making it
possible to define which functions aren't available (the GNU fileutils
will compile quite nicely, once a sufficient number of
{\tt-DXXX\_MISSING} flags are added to the {\tt Makefile}.)

\subsection{Programs already ported}

These programs have already been ported to \Linux:
\begin{itemize}
\item GNU cc (gcc, cc1, cpp)
\item GNU assembler (as386)
\item GNU binutils (ld, ar, nm, size, strip, ranlib)
\item GNU compress (16-bit)
\item GNU tar
\item GNU make
\item GNU bash (Bourne Again SHell)
\item GNU sed
\item GNU bison (yacc-lookalike)
\item GNU awk
\item GNU fileutils (ls, cp, rm, mkdir, rmdir, tail etc)
\item less
\item uemacs
\end{itemize}

All of the above sources can be found on \nic\ (mostly in
{\tt`/pub/gnu'}), and most of the \Linux-binaries can be found in the
directory \dir. All of these programs compiled without changes, even
though gcc (cc1) has some enhancements of my own. Mail me for
diffs/sources, but try first to compile them yourself.

Additionally I have reports that GNU diff compiles cleanly and works.

\section{Technical help}

\Linux\ currently has a mailing-list, which you can subscribe to by
mailing to the address: {\tt\bf Linux-activists-request@niksula.hut.fi}
and asking to be included into the list.  You can then ask questions by
mailing them to {\tt\bf Linux-activists@niksula.hut.fi}, which will
duplicate your question/answer/whatever and send them to all persons on
the list. 

Note the difference between {\tt Linux-activists} and
{\tt Linux-activists-request} --- the first sends mail to all persons on
the list, the second is used only to subscribe and unsubscribe from the
list.

Naturally, you can also mail me directly at {\tt\bf
torvalds@kruuna.helsinki.fi}.  I'll try to answer all questions within a
day or two.

Although \nic\ will probably be kept reasonably up-to-date, there are a
few problems with it (ie, I cannot personally get to the files, but have
to go through a couple of persons). Thus people on the mailing-list will
get patches/binaries faster if they ask for them.

\section{Thanks}

I'd like to thank the academy \dots

Seriously, this system never would have seen the light of day or would
have been much worse without the help of some others.  Bruce Evans
helped me find the places needed to be changed in order for {\tt gcc} to
correctly handle floating point, and came with a lot of useful
ideas/suggestions (and his Minix-386 was used to build the system). 
Also, Earl Chew's {\tt estdio} package was used for the standard
IO-library.  More freely distributable packages like this!

Alain W Black and Richard Tobin made the gcc for Minix, without which I
couldn't have compiled the thing.  GNU made most of the programs I use
under \Linux.  Alfred Leung sent the US keyboard patches. 

PS. ``Thanks'' to {\tt wirzeniu@kruuna.helsinki.fi} for his
``constructive'' criticism and ``witty'' comments. He was also my first
$\alpha$-tester, and should be given a medal for courage.

\end{document}
