\documentstyle{article}

\parindent=0pt
\parskip=4pt

% Okay, what follows is more TeX than LaTeX ...

\def\hditem#1{\hbox to 1.2in{#1\hfil}}
\def\boottwo#1{$$
  \bf
  \begin{tabular}{|ll|}
    \hline
    \hditem{\rm Master Boot Record} & \hditem{\rm Operating system} \\
    \hline
    #1 \\
    \hline
  \end{tabular}
  $$}

\def\bootthree#1{$$
  \bf
  \begin{tabular}{|lll|}
    \hline
    \hditem{\rm Master Boot Record} & \hditem{\rm Boot sector} &
      \hditem{\rm Operating system} \\
    \hline
    #1 \\
    \hline
  \end{tabular}
  $$}

\def\bootfour#1{$$
  \bf
  \begin{tabular}{|llll|}
    \hline
    \hditem{\rm Master Boot Record} & \hbox to 1.1in{\rm Boot sector\hfil} &
      \hditem{\rm Operating systems} & \hbox to 0.4in{\hfil} \\
    \hline
    #1 \\
    \hline
  \end{tabular}
  $$}
\def\sep{\rightarrowfill &}
\def\empty{&}
\def\branch{\hfill$\longrightarrow$ &}
\def\cont{---\,$\cdots$}
\def\key#1{#1}

\begin{document}

\title{LILO \\
  Generic boot loader for Linux}
\author{Werner Almesberger \\
  {\tt almesber@nessie.cs.id.ethz.ch}}

\maketitle
\setcounter{tocdepth}{1}
%\tableofcontents

LILO is a versatile boot loader for Linux. It does not depend on a specfic
file system, can boot Linux boot images and unstripped Linux kernels from
floppy disks and from hard disks and can even boot other operating
systems\footnote{MS-DOS, DR DOS, OS/2, 386BSD, $\ldots$}.

Up to sixteen different boot images can be selected at boot time. The root and
swap device can be set independently for each of them. LILO can even be
used as the master boot record.

This document introduces the basics of disk organization and booting,
continues with an overview of common boot techniques and finally describes
installation and use of LILO in greater detail.


\section{Disk organization}

When designing a boot concept, it is important to understand all the
subtleties of how MS-DOS organizes disks. The most simple case are
floppy disks. They consist of a boot sector, some administrative
data (FAT or super block, etc.) and the data area. Because that
administrative data is irrelevant as far as booting is concerned, it is
added to the data area for simplicity.

$$
\begin{tabular}{|c|c|}
  \hline
  Boot sector & \hbox to 1.5in{\hfil} \\
  \cline{1-1}
  \multicolumn{2}{|c|}{} \\
  \multicolumn{2}{|c|}{Data area} \\
  \multicolumn{2}{|c|}{} \\
  \hline
  \end{tabular}
$$

The entire disk appears as one device (i.e. {\tt /dev/fd0}) on Linux.

The MS-DOS boot sector has the following structure:

$$
\begin{tabular}{r|c|}
  \cline{2-2}
  \tt 0x000 & Jump to the program code\\
  \cline{2-2}
  \tt 0x003 & \\
  & Disk parameters \\
  & \\
  \cline{2-2}
  \tt 0x02C/0x03E & \\
  & Program code \\
  & \\
  & \\
  \cline{2-2}
  \tt 0x1FE & Magic number (0xAA55)\\
  \cline{2-2}
\end{tabular}
$$

LILO uses a similar boot sector, but it does not contain the disk
parameters part. This is no problem for Minix or EXT file systems, because
they don't look at the boot sector, but putting a LILO boot sector on an
MS-DOS file system makes it inaccessible for MS-DOS.

Hard disks are organized in a more complex way than floppy disks. They
contain several data areas called partitions. Up to four so-called
primary partitions can exist on an MS-DOS hard disk. If more partitions
are needed, a primary partition is used as an extended partition that
contains several logical partitions.

The first sector of each hard disk contains a partition table and an
extended partition and {\bf each} logical partition contains a partition
table too.\footnote{Is it legal to have more than one extended partition ?
Linux appears to be able to handle this, but is DOS ?}

$$
\begin{tabular}{|l|l|l|}
  \hline
  \multicolumn{3}{|l|}{Partition table\hbox to 2in{\hfil\tt /dev/hda~}} \\
  \cline{2-3}
  & \multicolumn{2}{l|}{Partition 1\hfill {\tt /dev/hda1}} \\
  & \multicolumn{2}{l|}{} \\
  \cline{2-3}
  & \multicolumn{2}{l|}{Partition 2\hfill {\tt /dev/hda2}} \\
  & \multicolumn{2}{l|}{} \\
  \hline
\end{tabular}
$$

The entire disk can be accessed as {\tt /dev/hda}, {\tt /dev/hdb},
{\tt /dev/sda}, etc. The primary partitions are {\tt /dev/hda1 $\ldots$
/dev/hda4}.

$$
\begin{tabular}{|l|l|l|}
  \hline
  \multicolumn{3}{|l|}{Partition table\hbox to 2in{\hfil\tt /dev/hda~}} \\
  \cline{2-3}
  & \multicolumn{2}{l|}{Partition 1\hfill {\tt /dev/hda1}} \\
  & \multicolumn{2}{l|}{} \\
  \cline{2-3}
  & \multicolumn{2}{l|}{Partition 2\hfill {\tt /dev/hda2}} \\
  & \multicolumn{2}{l|}{} \\
  \cline{2-3}
  & \multicolumn{2}{l|}{Extended partition\hfill {\tt /dev/hda3}} \\
  \cline{3-3}
  & & Extended partition table \\
  \cline{3-3}
  & & Partition 3\hfill {\tt /dev/hda5}\\
  & & \\
  \cline{3-3}
  & & Extended partition table \\
  \cline{3-3}
& & Partition 4\hfill {\tt /dev/hda6}\\
  & & \\
  \hline
\end{tabular}
$$

This hard disk has two primary partitions and an extended partition
that contains two logical partitions. They are accessed as
{\tt /dev/hda5 $\ldots$}

Note that the partition tables of logical partitions are not accessible
as the first blocks of some devices, while the main partition table,
all boot sectors and the partition tables of extended partitions are.

Partition tables are stored in partition boot sectors. Only the partition
boot sector of the entire disk is usually used as a boot sector. It is
also frequently called the master boot record (MBR).

$$
\begin{tabular}{r|c|}
  \cline{2-2}
  \tt 0x000 & \\
  & Program code \\
  & \\
  & \\
  \cline{2-2}
  \tt 0x1BE & Partition table \\
  & \\
  \cline{2-2}
  \tt 0x1FE & Magic number (0xAA55) \\
  \cline{2-2}
\end{tabular}
$$

The LILO boot sector is designed to be usable as a partition boot sector.
Therefore, the LILO boot sector can be stored at the following locations:

\begin{itemize}
  \item boot sector of a Linux floppy disk. ({\tt /dev/fd0}, $\ldots$)
  \item MBR of the first hard disk. ({\tt /dev/hda}, $\ldots$)
  \item boot sector of a Linux primary partition on the first hard
    disk. ({\tt /dev/hda1}, $\ldots$)
  \item partition boot sector of an extended partition on the first hard disk.
    ({\tt /dev/hda1}, $\ldots$)\footnote{Most FDISK-type programs don't
      believe in booting from an extended partition and refuse to
      activate it. LILO is accompanied by a simple program that doesn't
      have this restriction.}
\end{itemize}

It {\bf can't} be stored at any of the following locations:

\begin{itemize}
  \item boot sector of a non-Linux floppy disk of primary partition.
  \item a Linux swap partition.
  \item boot sector of a logical partition in an extended partition.
  \item on the second hard disk. (Unless for backup installations or
    if the current first disk will be removed or disabled.)
\end{itemize}


\section{Booting basics}

When booting from a floppy disk, the first sector of the disk, the so-called
boot sector, is loaded. That boot sector contains a small program that loads
the respective operating system. MS-DOS boot sectors also contain 
a data area, where the disk parameters (number of sectors, number of
heads, etc.) are stored.

When booting from a hard disk, the very first sector of that disk, the
so-called Master boot record (MBR) is loaded. This sector contains a
loader program and the partition table of the disk. The loader program
usually loads the boot sector, as if the system was booting from a floppy.

Note that there is no functional difference between the MBR and the boot
sector other than that the MBR contains the partition information but
doesn't contain any (MS-DOS) disk parameter information.

The first 446 (0x1BE) bytes of the MBR are used by the loader program.
They are followed by the partition table, with a length of 64 (0x40)
bytes. The last two bytes contain a magic number that is sometimes used to
verify that a given sector really is a boot sector.

There is a large number of possible boot configurations. The most common
ones are described in the following sections.


\subsection{MS-DOS alone}

\bootthree{DOS-MBR \sep MS-DOS \sep COMMAND.COM}

This is what usually happens when MS-DOS boots from a hard disk: the DOS-MBR
determines the active partition and loads the MS-DOS boot sector. This boot
sector loads MS-DOS and finally passes control to {\tt COMMAND.COM}. (This is
greatly simplified.)


\subsection{BOOTLIN}

\bootfour{DOS-MBR \sep MS-DOS \sep COMMAND.COM \empty \\
  \empty \branch BOOTLIN \sep Linux}

A typical BOOTLIN setup: everything happens like when booting MS-DOS, but in
{\tt CONFIG.SYS},
BOOTLIN is invoked. This approach has the pleasent property that no boot
sectors have to be altered.

Installation:
\begin{itemize}
  \item boot Linux.
  \item copy a bootable kernel image to your MS-DOS partition.\footnote{%
    With Mtools or the MS-DOS FS.}
  \item install {\tt BOOT.SYS} and {\tt BOOTLIN.SYS} on your MS-DOS
    partition and add them to your {\tt CONFIG.SYS}.
    (The READMEs describe how this is done.)
  \item reboot.
\end{itemize}

Deinstallation:
\begin{itemize}
  \item remove {\tt BOOT.SYS} and {\tt BOOTLIN.SYS} from your {\tt CONFIG.SYS}.
\end{itemize}


\subsection{LILO started by DOS-MBR}

\bootthree{DOS-MBR \sep LILO \sep Linux \\
  \branch {\rm other OS} \empty}

This is a ``safe'' LILO setup: LILO is booted by the DOS-MBR. No other boot
sectors have to be touched. If the other OS (or one of them, if there are
several of them) should be booted, the other partition has to be marked
``active'' with (e)fdisk.

Installation:
\begin{itemize}
  \item install LILO with its boot sector on the Linux partition.
  \item use (e)fdisk to mark that partition active.
  \item reboot.
\end{itemize}

Deinstallation:
\begin{itemize}
  \item mark a different partition active.
  \item install whatever should replace LILO or Linux.
\end{itemize}


\subsection{Several alternate branches}

\bootfour{DOS-MBR \sep MS-DOS \sep COMMAND.COM \empty \\
  \empty \branch BOOTLIN \sep Linux \\
  \branch LILO \sep Linux \empty \\
  \empty \branch MS-DOS \cont \empty}

An extended form of the above setup: the MBR is not changed and both branches
can either boot Linux or MS-DOS. (LILO could also boot any other
operating system.)


\subsection{LILO started by BOOTACTV}

\bootthree{BOOTACTV \sep LILO \sep Linux \\
  \branch {\rm other OS} \empty}

Here, the MBR is replaced by BOOTACTV (or any other interactive boot
partition selector) and the choice between Linux and the
other operating system can be made at boot time. This approach should be
used if LILO fails to boot the other operating system(s).\footnote{%
And the author would like to be notified if booting the other operating
system(s) doesn't work with LILO, but if it works with an other boot partition
selector.}

Installation:
\begin{itemize}
  \item boot Linux.
  \item make a backup copy of your MBR on a floppy disk, e.g. \\
    \verb"dd if=/dev/hda of=/fd/MBR bs=512 count=1"
  \item install LILO with the boot sector on the Linux partition.
  \item install BOOTACTV as the MBR, e.g. \\
    \verb"dd if=bootactv.bin of=/dev/hda bs=446 count=1"
  \item reboot.
\end{itemize}

Deinstallation:
\begin{itemize}
  \item boot Linux.
  \item restore the old MBR, e.g. \\
    \verb"dd if=/MBR of=/dev/hda bs=446 count=1"
\end{itemize}

If replacing the MBR appears undesirable and if a second Linux partition
exists (e.g. {\tt /usr}, {\bf not} a swap partition), BOOTACTV can be merged
with
the partition table and stored as the ``boot sector'' of that partition.
Then, the partition can be marked active to be booted by the DOS-MBR.

Example:
\begin{verbatim}
# dd if=/dev/hda of=/dev/hda3 bs=512 count=1
# dd if=bootactv.bin of=/dev/hda3 bs=446 count=1
\end{verbatim}

Warning: whenever the disk is re-partitioned, the merged boot sector on that
``spare'' Linux partition has to be updated too.


\subsection{Shoelace started by BOOTACTV}

\bootthree{BOOTACTV \sep Shoelace \sep Linux \\
  \branch {\rm other OS} \empty}

Shoelace, LILO's predecessor, can be started by BOOTACTV as well, of course.
The same indirection as outlined above is possible. There are probably many
other ways to install Shoelace.


\subsection{LILO alone}

\boottwo{LILO \sep Linux \\
  \branch {\rm other OS}}

LILO can also take over the entire boot procedure. If installed as the MBR,
LILO is responsible for either booting Linux or any other OS. This approach
has the disadvantage, that the old MBR is overwritten and has to be restored
(either from a backup copy, with \verb"FDISK /MBR" on MS-DOS 5.0 or by
overwriting
it with BOOTACTV) if Linux should ever be removed from the system.

You should verify that LILO is able to boot your other operating system(s)
before relying on this method.

Installation:
\begin{itemize}
  \item boot Linux.
  \item make a backup copy of your MBR on a floppy disk, e.g. \\
    \verb"dd if=/dev/hda of=/fd/MBR bs=512 count=1"
  \item install LILO with its boot sector as the MBR.
  \item reboot.
\end{itemize}

Deinstallation:
\begin{itemize}
  \item boot Linux.
  \item restore the old MBR, e.g. \\
    \verb"dd if=/fd/MBR of=/dev/hda bs=446 count=1"
\end{itemize}

If you've installed LILO to be the master boot record, you have to
explicitly specify the boot sector when updating the map. Otherwise, it
will try to use the boot sector of your current root partition, which
may even work, but will leave your system unbootable.


\subsection{Special names}

The following names have been used to describe boot sectors or parts of
operating systems:

\begin{description}
  \item[\bf DOS-MBR] is the original MS-DOS MBR. It scans the partition
    table for a partition that is marked ``active'' and loads the boot
    sector of that partition. Programs like MS-DOS' fdisk, Owen Le Blanc's
    fdisk for Linux (on MCC-interim or 0.97 rootimage, or named efdisk on
    the 0.96 rootimage) or activate (accompanies LILO)
    can change the active marker in the partition table.
  \item[\bf MS-DOS] denotes the MS-DOS boot sector that loads the other parts
    of the system ({\tt IO.SYS}, etc.).
  \item[\bf COMMAND.COM] is the standard command interpreter of MS-DOS.
  \item[\bf BOOTLIN] is a program that loads a Linux boot image from an
    MS-DOS partition into memory and executes it. It is usually invoked
    from {\tt CONFIG.SYS} and used in conjunction with a {\tt CONFIG.SYS}
    configuration switcher, like BOOT.SYS.\footnote{BOOTLIN is
      available for anonymous FTP from \\
      {\tt tsx-11.mit.edu:/pub/linux/INSTALL/bootlin4.zip} or \\
      {\tt nic.funet.fi:/pub/OS/Linux/tools/bootlin.zip}, BOOT.SYS is
      available for anonymous FTP from
      {\tt nic.funet.fi:/pub/OS/Linux/tools/boot142.zip} or \\
      {\tt wuarchive.wustl.edu:/mirrors/msdos/sysutl/boot142.zip}.}
  \item[\bf LILO] can either load a Linux kernel or the boot sector of any
    other operating system. It consists of a first stage boot sector that
    loads the remaining parts of LILO from various locations.
  \item[\bf BOOTACTV] permits interactive selection of the partition from
    which the boot sector should be read. If no key is pressed within a
    given interval, the partition marked active is booted. BOOTACTV is
    included in the pfdisk package. There are also several similar
    programs, like PBOOT and OS-BS.\footnote{pfdisk is available for
      anonymous FTP from \\
      {\tt tsx-11.mit.edu:/pub/linux/INSTALL/pfdisktc.zip} or \\
      {\tt nic.funet.fi:/pub/OS/Linux/tools/pfdisk.tar.Z}. PBOOT can be
      found at the same sites in the same directories.}
  \item[\bf Shoelace] is a different boot loader for Linux. It is
    functionally similar to LILO, but it can only use the Minix file system.
\end{description}


\section{Choosing the ``right'' boot concept}

Although LILO can be installed in many different ways, the choice is
usually limited by the present setup and therefore there is
typically only a small number of configurations which fit naturally
into an existing system.

In all examples, the names of the AT-type hard disk devices
({\tt /dev/hda$\ldots$}) are used. Everything applies to SCSI disks
({\tt /dev/sda$\ldots$}) too.


\subsection{One disk, Linux on a primary partition}

If at least one primary partition of the first hard disk is used as a
Linux file system ({\tt /}, {\tt /usr}, etc. but {\bf not} for a swap
partition), the LILO boot sector should be stored on that partition
and it should be booted by the original master boot record or by a
program like BOOTACTV.

$$
  \begin{tabular}{r|c|c|}
    \cline{2-3}
    & \multicolumn{2}{|l|}{MBR\hbox to 1.3in{\hfill\tt /dev/hda~}} \\
    \cline{3-3}
    & & MS-DOS\hfill\tt /dev/hda1 \\
    \cline{3-3}
    $\rightarrow$ & & Linux {\tt /}\hfill\tt /dev/hda2 \\
    \cline{2-3}
  \end{tabular}
$$

A typical {\tt /etc/lilo/install} file would look like this:
\begin{verbatim}
/etc/lilo/lilo -c -i /etc/lilo/boot.b \
    /linux \
    /linux.backup \
    msdos=/etc/lilo/chain.b+/dev/hda1@/dev/hda
\end{verbatim}


\subsection{One disk, Linux on an extended partition}

If no primary partition is available for Linux, but at least one logical
partition of an extended partition on the first hard disk contains a
Linux file system, the LILO boot sector should be stored in the partition
sector of the extended partition and it should be booted by the original
master boot record or by a program like BOOTACTV.

$$
  \begin{tabular}{r|c|c|c|}
    \cline{2-4}
    & \multicolumn{3}{|l|}{MBR\hbox to 1.3in{\hfill\tt /dev/hda~}} \\
    \cline{3-4}
    & & \multicolumn{2}{|l|}{MS-DOS\hfill\tt /dev/hda1} \\
    \cline{3-4}
    $\rightarrow$ & & \multicolumn{2}{|l|}{Extended\hfill\tt /dev/hda2} \\
    \cline{4-4}
    & & & Linux\hfill\tt /dev/hda5 \\
    \cline{4-4}
    & & & $\ldots$\hfill\tt /dev/hda6 \\
    \cline{2-4}
  \end{tabular}
$$

A typical {\tt /etc/lilo/install} file for this configuration would look
like this:
\begin{verbatim}
/etc/lilo/lilo -b /dev/hda2 -c -i /etc/lilo/boot.b \
    /linux \
    /linux.backup \
    msdos=/etc/lilo/chain.b+/dev/hda1@/dev/hda
\end{verbatim}


\subsection{Two disks, Linux (at least partially) on the first disk}

This case is equivalent to the configurations, where only one disk
is in the system. The Linux boot sector resides on the first hard
disk and the second disk is used later.

Only the location of the boot sector matters -- everything
else ({\tt /etc/lilo/boot.b},
{\tt /etc/lilo/map}, the root file system, a swap partition, other
Linux file systems, etc.) can be located anywhere on the second disk.


\subsection{Two disks, Linux on the second disk, the first disk has an
  extended partition}

If there is no Linux partition on the first disk, but there is an
extended partition, the LILO boot sector can be stored in the partition
sector of the extended partition and it should be booted by the original
master boot record or by a program like BOOTACTV.

$$
  \begin{tabular}{r|c|c|c|c|c|c|}
    \multicolumn{1}{r}{}
    & \multicolumn{3}{c}{\bf First disk} &
      \multicolumn{1}{r}{\qquad}
      & \multicolumn{2}{c}{\bf Second disk} \\
    \cline{2-4}\cline{6-7}
    & \multicolumn{3}{|l|}{MBR\hbox to 1.3in{\hfill\tt /dev/hda~}} &
      & \multicolumn{2}{|l|}{MBR\hbox to 1.3in{\hfill\tt /dev/hdb~}} \\
    \cline{3-4}\cline{7-7}
    & & \multicolumn{2}{|l|}{MS-DOS\hfill\tt /dev/hda1} &
      & & Linux\hfill\tt /dev/hdb1 \\
    \cline{3-4}\cline{7-7}
    $\rightarrow$ & & \multicolumn{2}{|l|}{Extended\hfill\tt /dev/hda2} &
      & & $\ldots$\hfill\tt /dev/hdb2 \\
    \cline{4-4}
    & & & $\ldots$\hfill\tt /dev/hda5 & & & \\
    \cline{4-4}
    & & & $\ldots$\hfill\tt /dev/hda6 & & & \\
    \cline{2-4}\cline{6-7}
  \end{tabular}
$$

A typical {\tt /etc/lilo/install} file for this configuration would look
like this:
\begin{verbatim}
/etc/lilo/lilo -b /dev/hda2 -c -i /etc/lilo/boot.b \
    /linux \
    /linux.backup \
    msdos=/etc/lilo/chain.b+/dev/hda1@/dev/hda
\end{verbatim}

The program activate, that accompanies LILO, has to be used to set the
active marker on an extended partition, because MS-DOS' fdisk and (e)fdisk
refuse to do that. (Which is generally a good idea.)


\subsection{Two disks, Linux on the second disk, the first disk has no
  extended partition}

If there is neither a Linux partition nor an extended partition on the first
disk, where a LILO boot sector could be stored, then there's only one place
left: the master boot record.

In this configuration, LILO boots all other operating systems too.

$$
  \begin{tabular}{r|c|c|c|c|c|}
    \multicolumn{1}{r}{}
    & \multicolumn{2}{c}{\bf First disk} &
      \multicolumn{1}{r}{\qquad}
      & \multicolumn{2}{c}{\bf Second disk} \\
    \cline{2-3}\cline{5-6}
    $\rightarrow$ & \multicolumn{2}{|l|}{MBR\hbox to 1.3in{
      \hfill\tt /dev/hda~}} &
      & \multicolumn{2}{|l|}{MBR\hbox to 1.3in{\hfill\tt /dev/hdb~}} \\
    \cline{3-3}\cline{6-6}
    & & MS-DOS\hfill\tt /dev/hda1 &
      & & Linux\hfill\tt /dev/hdb1 \\
    \cline{3-3}\cline{6-6}
    & & $\ldots$\hfill\tt /dev/hda2 &
      & & $\ldots$\hfill\tt /dev/hdb2 \\
    \cline{2-3}\cline{5-6}
  \end{tabular}
$$

You should back up your old MBR before installing LILO and verify that
LILO is able to boot your other operating system(s) before relying on
this approach.

A typical {\tt /etc/lilo/install} file for this configuration would look
like this:
\begin{verbatim}
/etc/lilo/lilo -b /dev/hda -c -i /etc/lilo/boot.b \
    /linux \
    /linux.backup \
    msdos=/etc/lilo/chain.b+/dev/hda1@/dev/hda
\end{verbatim}


\section{Technical overview}

This section contains a description of several internals of LILO. It is
not necessary to understand them in order to install LILO.


\subsection{Load sequence}

The boot sector is loaded by the ROM-BIOS at address 0x07C00. It moves
itself to address 0x90000, sets up the stack (growing downwards from
0x92000 to 0x91000), loads the secondary boot loader at address
0x92000 and transfers control to it. It displays an ``L'' after moving
itself and an ``I'' before starting the secondary boot loader.

The secondary boot loader loads the descriptor table at 0x92E00 and checks
for user input. If either the default is used or if the user has specified
an alternate image, the setup part of that image is loaded at 0x90200 and
the kernel part is loaded at 0x10000. During that load operation, the sectors
of the map file are loaded at 0x93000.

If the loaded image is a traditional boot image, control is transferred to
its setup code. If it is an unstripped kernel, its BSS is zeroed first.
If a different operating system is booted, things are a bit more difficult:
the chain loader is loaded at 0x90200 and the boot sector of the other OS
is loaded at 0x90400. The chain loader moves the partition table (loaded at
0x903BE as part of the chain loader) to 0x00600 and the boot sector to
0x07C00. After that, it passes control to the boot sector.

The secondary boot loader displays an ``L'' after being started and an ``O''
after loading the descriptor table.

$$
\begin{tabular}{l|c|l}
  \cline{2-2}
  \tt 0x00000 & & \\
  \cline{2-2}
  \tt 0x00600 & Partition table & 64 bytes \\
  \cline{2-2}
  \tt 0x00640 & & \\
  \cline{2-2}
  \tt 0x07C00 & Boot load area & 512 bytes \\
  \cline{2-2}
  \tt 0x07E00 & & \\
  \cline{2-2}
  & & \\
  \tt 0x10000 & Kernel & 320 kB \\
  & & \\
  & & \\
  \cline{2-2}
  \tt 0x60000 & & \\
  & & \\
  & & \\
  \cline{2-2}
  \tt 0x90000 & Primary boot loader & 512 bytes \\
  \cline{2-2}
  \tt 0x90200 & Setup (kernel) & 2 kB \\
  \cline{2-2}
  \tt 0x90A00 & & \\
  \cline{2-2}
  \tt 0x91000 & Stack & 4 kB \\
  \cline{2-2}
  \tt 0x92000 & Secondary boot loader & 3.5 kB \\
  \cline{2-2}
  \tt 0x92E00 & Descriptor table & 512 bytes \\
  \cline{2-2}
  \tt 0x93000 & Map load area & 512 bytes \\
  \cline{2-2}
  \tt 0x93200 & Scratch space & 51.5 kB \\
  & & \\
  \cline{2-2}
  \multicolumn{3}{l}{\tt 0xA0000} \\
\end{tabular}
$$


\subsection{File references}

This section describes the references among files involved in the boot
procedures.

$$
  \input bootloader
$$

The boot sector contains the primary boot loader, the address of the
descriptor table sector and the addresses of the sectors of the secondary
boot loader. The boot sector is copied from {\tt boot.b}.

The primary boot loader can store up to four sector addresses of the
secondary boot loader.

$$
  \input map
$$

The map file consists of sections and of special data sectors. Each section
spans an integral number of disk sectors and contains addresses of sectors
of other files.\footnote{There are two exceptions: 1. If a ``hole'' is being
covered, the address of the zero sector is used. This sector is part of the
map file. 2. When booting a different operating system, the first sector is
the merged chain loader that has been written to the map file before that
section.} The last address slot of each sector is either unused (if the map
ends in this sector) or contains the address of the next sector in the
section.

The two sectors at the beginning of the map file are special: the first
sector contains the boot image descriptor table and the second sector
is filled with zero bytes. This sector is mapped whenever a file contains
a ``hole''.

$$
  \input image
$$

A traditional boot image consists simply of a sequence of sectors that are
loaded. Images that are loaded from a device are treated exactly the same
way as images that are loaded from a file.

The first sector of the boot image contains the floppy boot sector
and is not mapped.

$$
  \input unstripped
$$

Unstripped kernels consist of the setup part and of the kernel file. The
descriptor also contains information about the start and the size of the
BSS segment. The boot loader clears BSS before starting the kernel.

$$
  \input other
$$

When booting another operating system, the chain loader ({\tt chain.b}) is
merged with the partition table\footnote{If the partition table is omitted,
that area is filled with zero bytes.} and written into the map file. The
map section of this boot image starts after that sector and contains only
the address of the loader sector and of the boot sector of the other
operating system.


\section{The boot prompt}

Immediately after it's loaded, LILO checks, whether one of the following
is happening:

\begin{itemize}
  \item any of the \key{Shift}, \key{Control} or \key{Alt} keys is being
    pressed.
  \item \key{CapsLock} or \key{ScrollLock} is set.
\end{itemize}

If this is the case, LILO displays the \verb"boot:" prompt and waits for
the name of a boot image. Otherwise, it boots the default boot image\footnote{%
The default boot image is either the first boot image or the image that
has been selected at the boot prompt.}
or -- if a
delay has been specified -- waits for one of the listed activities
until that amount of time has passed.

At the boot prompt, the name of the image to boot can be entered. Typing
errors can be corrected with the keys \key{BackSpace}, \key{Delete},
\key{Ctrl U} and \key{Ctrl X}. A list of known images can be obtained by
pressing \key{?} (on the US keyboard) or \key{Tab}.

If \key{Enter} is pressed and no file name has been entered, the default
image is booted.


\section{Map installer}

The map installer program {\tt /etc/lilo/lilo} updates the boot sector
and creates the map file. It is usually run from the shell script
{\tt /etc/lilo/install}. If the map installer detects an error, it
terminates immediately and does not touch the boot sector and the map
file.

Whenever LILO updates a boot sector, the original boot sector is copied
to {\tt /etc/lilo/boot.{\it number\/}}, where {\it number\/} is the
hexadecimal device number. If such a file already exists, no backup
copy is made.

LILO may create some device special files in your {\tt /tmp} directory that
are not removed if an error occurs. They are named
{\tt /tmp/dev.{\it number}}.


\subsection{Command-line arguments}

The LILO map installer accepts the following command-line options:

\begin{description}
  \item[\tt -b \it boot\_device]~ \\
    Sets the name of the device that contains the boot sector. If {\tt -b}
    is omitted, the boot sector is read from (and possibly written to) the
    device that is currently mounted as root. A BIOS device code can be
    specified.
  \item[\tt -c]~ \\
    Tries to merge read requests for adjacent sectors into a single read
    request. This drastically reduces load time and keeps the map
    smaller. Using {\tt -c} is especially recommended when booting from
    a floppy disk.
  \item[\tt -i \it boot\_sector]~ \\
    Install the specified file as the new boot sector. If {\tt -i} is omitted,
    the old boot sector is modified. A BIOS device code can be specified.
    {\tt -i} is usually a permanent part of the invocation of the map installer
    in {\tt /etc/lilo/install}.
  \item[\tt -m \it map\_file]~ \\
    Specifies the location of the map file. If {\tt -m} is omitted, a file
    {\tt /etc/lilo/map} is used. A BIOS device code can be specified.
  \item[\tt -r \it root\_directory]~ \\
    Chroot to the specified directory before doing anything else. This is
    useful when running the map installer while the normal root file system
    is mounted somewhere else, e.g. when recovering from an installation
    failure with a bootimage/rootimage.\footnote{I.e. if your root partition
      is mounted on {\tt /mnt}, you can update the map by simply running
      {\tt /mnt/etc/lilo/install} with the argument {\tt -r /mnt}. If you're
      normally using the default boot sector, you have to specify it
      explicitly in this case: {\tt -b /dev/{\it device\_name}}. So the
      complete command may be something like this: \\
      \verb"/mnt/etc/lilo/install -r /mnt -b /dev/hda1" \\
      You also have to set the environment variable {\tt ROOT} before running
      the update script, e.g.: \\
      \verb"export ROOT=/mnt"}
  \item[\tt -s \it backup\_file]~ \\
    Copy the original boot sector to {\it backup\_file\/} (which may also be
    a device, e.g. {\tt /dev/null}) instead of
    {\tt /etc/lilo/boot.{\it number}}
  \item[\tt -S \it backup\_file]~ \\
    Like {\tt -S}, but overwrite an old backup copy if it exists.
  \item[\tt -t]~ \\
    Test only. This performs the entire installation procedure except
    replacing the map file and writing the modified boot sector. This
    can be used in conjunction with the {\tt -v} option to verify that LILO
    will use sane values.
  \item[\tt -v]~ \\
    Turns on lots of progress reporting. Repeating {\tt -v} will turn on more
    reporting. ({\tt -v -v -v -v} is the highest verbosity level and displays
    all mappings.)
\end{description}

If no image files are specified, the currently mapped files are listed. Only
the options {\tt -m}, {\tt -v} and {\tt -r} can be used in this mode.

If at least one file name is specified, a new map is created for those files
and they are registered in the boot sector. If root or swap devices have been
set in the images, they are copied into the descriptors in the boot sector.
If no root device has been set\footnote{Or if this is not a traditional
boot image.}, the current root device is used. The root and
swap devices can be overridden by appending them to the image specification,
e.g.

$$
  \verb"lilo "\underbrace{\strut\verb"foo"}_{\rm image}\verb","%
\underbrace{\strut\verb"/dev/hda1"}_{\rm root}\verb","%
\underbrace{\strut\verb"0x302"}_{\rm swap}
$$

Either numbers or device names can be used.

It is perfectly valid to use different root/swap settings for the same
image, because LILO stores them in the image descriptors and not in the
images themselves. Example:

\begin{verbatim}
    lin-hd=/linux,/dev/hda2
    lin-fd=/linux,/dev/fd0
\end{verbatim}

The image files can reside on any media that is accessible at boot time.
There's no need to put them on the root device, although this certainly
doesn't hurt.

If LILO doesn't guess the correct BIOS device code, it can be specified by
appending a colon and the code to the file name, e.g. \verb"/linux:0x80"%
\footnote{This is typically used to install LILO on a second disk that
will be usd as the first disk later.}

LILO uses the first file name (without its path) of each image specification
to identify that image. A different name can be used by prefixing the
specification with {\it label\/}{\tt =}, e.g.

\begin{verbatim}
    msdos=/etc/lilo/chain.b+/dev/sda1@/dev/sda
\end{verbatim}


\subsection{Boot image types}

LILO can boot the following types of images:
\begin{itemize}
  \item ``traditional'' boot images from a file.
  \item ``traditional'' boot images from a block device. (I.e. a floppy
    disk.)
  \item unstripped kernel executables.
  \item the boot sector of some other OS.
\end{itemize}

The image type is determined by the syntax that is used for the image
specification.


\subsubsection{Booting ``traditional'' boot images from a file}

If defined, root and swap definitions are taken from the boot image.
The image is specified as follows:

$$\hbox{{\it file\_name}\big[\verb":"{\it BIOS\_code}\big]}$$

Example: \verb"/linux"


\subsubsection{Booting ``traditional'' boot images from a device}

Root and swap settings in the image are ignored. The range of sectors
that should be mapped, has to be specified. Either a range
({\it start\/\tt -\it end\/}) or a start and a distance
({\it start\/\tt +\it number\/}) have to be specified. If only
the start if specified, only that sector is mapped.

$$\hbox{{\it device\_name}\big[\verb":"{\it BIOS\_code}\big]\verb"#"%
{\it start}\big[\verb"-"{\it end}\big\vert\verb"+"{\it number}\big]}$$

Example: \verb"/dev/fd0#1+512"


\subsubsection{Booting unstripped kernel executables}

Unstripped kernel executables contain no root or swap device information.
The setup code of the kernel has also to be added to the kernel. First,
it has to be copied to a suitable place and its header has to be removed,
e.g.
\begin{verbatim}
    dd if=/usr/src/linux/boot/setup of=/etc/lilo/setup.b \
      bs=32 skip=1
\end{verbatim}
If this command is placed at the beginning of {\tt /etc/lilo/install},
{\tt setup.b} is automatically updated whenever that script is run
because anything else changes.

The image specification looks like this:

$$\hbox{{\it setup\_name}\big[\verb":"{\it BIOS\_code}\big]\verb"+"%
{\it kernel\_name}\big[\verb":"{\it BIOS\_code}\big]}$$

Example: \verb"/etc/lilo/setup.b+/usr/src/linux/tools/system"


\subsubsection{Booting a foreign OS}

LILO can even boot other operating systems, i.e. MS-DOS. This feature
is new and may not yet work totally reliably. (Reported to work with
PC-DOS 4.0, MS-DOS 5.0, DR-DOS 6.0, OS/2 2.0 and 386BSD.) To boot an other
operating
system, the name of a loader program, the device that contains the boot
sector and the device that contains the partition table have to be
specified:

$$\hbox{{\it loader}\verb"+"{\it boot\_dev}\big[\verb":"{\it BIOS\_code}\big]%
\verb"@"\big[{\it part\_dev}\big]}$$

Example: \verb"/etc/lilo/chain.b+/dev/hda2@/dev/hda"

The name of the device that contains the partition table can be omitted if
the respective operating system has other means to determine from which
partition it has been booted.
(E.g. MS-DOS stores the geometry of the boot disk or partition in its boot
sector.)

The boot sector is merged with the partition table and stored in the map file.

Currently, only the loader {\tt chain.b} exists. {\tt chain.b} simply
starts the specified boot sector.\footnote{The boot sector is loaded by
LILO's second boot loader before control is passed to the code of
{\tt chain.b}.}


\subsection{Disk parameter table}
\label{disktab}

For floppies and IDE disks (or MFM, RLL, ESDI, etc.), LILO can obtain the
disk geometry information from the kernel. Unfortunately, this is not
possible with SCSI disks. The file {\tt /etc/lilo/disktab} is used to describe
such disks. For each device ({\tt /dev/sda} $\rightarrow$ 0x800,
{\tt /dev/sda1} $\rightarrow$ 0x801, etc.),
the BIOS code, the disk geometry and the offset of the first sector of
that partition (measured in sectors) have to be specified, e.g.

\begin{verbatim}
# /etc/lilo/disktab  -  LILO disk parameter table
#
# This table contains disk parameters for SCSI disks and non-
# standard parameters of IDE disks. Parameters in disktab
# _always_ override auto-detected disk parameters.

# Dev.  BIOS    Secs/   Heads/  Cylin-  Part.
# num.  code    track   cylin.  ders    offset

0x800   0x80    32      64      631     0       # /dev/sda
0x801   0x80    32      64      631     32      # /dev/sda1
0x802   0x80    32      64      631     204800  # /dev/sda2
\end{verbatim}

(Those parameters are just a random example from my system. However, many
SCSI controllers
re-map the drives to 32 sectors and 64 heads. The number of cylinders
does not have to be exact, but it shouldn't be lower than the number of
effectively available cylinders.)

Note that the device number and the BIOS code have to specified as
hexadecimal numbers with the ``0x'' prefix.

The disk geometry parameters can be obtained by booting MS-DOS and
running the program {\tt DPARAM.COM} with the hexadecimal BIOS code of
the drive as its argument, e.g. \verb"dparam 0x80" for the first hard
disk. It displays the number of sectors per
track, the number of heads per cylinder and the number of cylinders.

The partition offset is printed by the Linux kernel when the SCSI disk
is detected at boot time. Example:

\begin{verbatim}
sd0 :
 part 1 start 32 size 204768 end 204799
 part 2 start 204800 size 1087488 end 1292287
\end{verbatim}

The first partition has an offset of 32 sectors, the second has an
offset of 204800 sectors.

Because many SCSI controllers don't support more than 1 GB when using
the BIOS interface, LILO can't access files that are located beyond the
1 GB limit of large SCSI disks and reports errors in these cases.


\section{Installation}

\subsection{First-time installation}

You have to run the 0.96c-pl1 kernel or any newer release. First, you
have to install the LILO files:

\begin{itemize}
  \item extract all files from {\tt lilo.{\it version}.tar.Z}
  \item run\quad{\tt make}\quad to compile and assemble all parts.
  \item run\quad{\tt make install}\quad to copy all LILO files to
     {\tt /etc/lilo}.
     {\tt /etc/lilo} should now contain the following files: {\tt boot.b},
     {\tt chain.b}, {\tt disktab} and {\tt lilo}.
\end{itemize}

If you want to use LILO on a SCSI disk, you have to determine the
parameters of your SCSI disk(s) and put them into the file
{\tt /etc/lilo/disktab}. See section \ref{disktab} for details.

The next step is to test LILO with the boot sector on a floppy disk:

\begin{itemize}
  \item insert a blank (but formatted) floppy disk into {\tt /dev/fd0}.
  \item chdir to {\tt /etc/lilo}.
  \item run {\tt ./lilo -b /dev/fd0 -i boot.b -v -v -v {\it kernel\_image(s)}}%
\footnote{If you've already installed LILO on your system, you might not want
  to overwrite your old map file.
  Use the {\tt -m} option to specify an alternate map file name.}
  \item reboot. LILO should now load its boot loaders from the floppy disk
    and then continue loading the kernel from the hard disk.
\end{itemize}

Now, your have to decide, which boot concept you want to use. Let's assume
you have a Linux partition on {\tt /dev/hda2} and you want to install your
LILO boot sector there. The DOS-MBR loads the LILO boot sector.

\begin{itemize}
  \item get a working bootimage and a rootimage. Verify that you can boot
    with this setup and that you can mount your Linux partition(s) with it.
  \item if the boot sector you want to overwrite with LILO is of any value
    (e.g. it's the MBR or it contains a boot loader you might want to use
    if you encounter problems with LILO), you should mount your rootimage
    and make a backup copy of your boot sector to a file on that floppy,
    e.g. \verb"dd if=/dev/hda of=/fd/boot_sector bs=512 count=1"
  \item create a shell script {\tt /etc/lilo/install} that installs and
    updates LILO on your hard disk, e.g.\\
\verb"#!/bin/sh" \\
\verb"$ROOT/etc/lilo/lilo "{\it all\_necessary\_options\/}%
\verb" -i /etc/lilo/boot.b $* \" \\
\verb"  "{\it kernel\_images}
  \item Now, you can check what LILO would do if you were about to install
    it on your hard disk: \\
\verb"/etc/lilo/install -v -v -v -t"
  \item If you need some additional boot utility (i.e. BOOTACTV), you should
    install that now.
  \item Run {\tt /etc/lilo/install} to install LILO on your hard disk.
  \item If you have to change the active partition, use (e)fdisk or
    activate to do that.
  \item Reboot.
\end{itemize}


\subsection{LILO update}

When updating to a new version of LILO, the initial steps are the same as
for a first time installation: extract all files, run {\tt make} to build
the executables and run {\tt make install} to move the files to
{\tt /etc/lilo}.

The old versions of {\tt boot.b} and {\tt chain.b} are automatically
renamed to {\tt boot.old} and {\tt chain.old}. This is done to ensure
that you can boot even if the installation procedure is not finished.
{\tt boot.old} and {\tt chain.old} can be deleted after the map file
is rebuilt.

Because the locations of {\tt boot.b} and {\tt chain.b} have changed
and because the map file format may be different too, you have to update
the boot sector and the map file. Run {\tt /etc/lilo/install} to do this.


\subsection{Kernel update}

Whenever any of the kernel files that are accessed by LILO is moved or
overwritten, the map has to be be re-built.\footnote{It is advisable to
keep a secondary, stable, boot image that can be booted if you forget
to update the map after a change to your usual boot image.} Run
{\tt /etc/lilo/install} to do this.

If the setup code has changed and if you're booting unstripped kernels,
you also have to update {\tt setup.b}. This should be done in
{\tt /etc/lilo/install}.

If you're frequently re-compiling the kernel, you should put an invocation of
{\tt /etc/lilo/install} into the kernel's top Makefile.

Example ({\tt $\ldots$/linux/Makefile}):

\begin{verbatim}
...
Image: boot/bootsect boot/setup tools/system tools/build
       cp tools/system system.tmp
       strip system.tmp
       tools/build boot/bootsect boot/setup system.tmp \
            $(ROOT_DEV) >/linux
       /etc/lilo/install
       rm system.tmp
       sync
...
\end{verbatim}

\end{document}
